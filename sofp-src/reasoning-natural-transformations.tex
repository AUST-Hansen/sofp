%% LyX 2.3.3 created this file.  For more info, see http://www.lyx.org/.
%% Do not edit unless you really know what you are doing.
\documentclass[english]{beamer}
\usepackage[T1]{fontenc}
\usepackage[latin9]{inputenc}
\setcounter{secnumdepth}{3}
\setcounter{tocdepth}{3}
\usepackage{babel}
\usepackage{amsmath}
\usepackage{amssymb}
\usepackage{stmaryrd}
\usepackage{wasysym}
\usepackage[all]{xy}
\ifx\hypersetup\undefined
  \AtBeginDocument{%
    \hypersetup{unicode=true,pdfusetitle,
 bookmarks=true,bookmarksnumbered=false,bookmarksopen=false,
 breaklinks=false,pdfborder={0 0 1},backref=false,colorlinks=true}
  }
\else
  \hypersetup{unicode=true,pdfusetitle,
 bookmarks=true,bookmarksnumbered=false,bookmarksopen=false,
 breaklinks=false,pdfborder={0 0 1},backref=false,colorlinks=true}
\fi

\makeatletter

%%%%%%%%%%%%%%%%%%%%%%%%%%%%%% LyX specific LaTeX commands.
%% Because html converters don't know tabularnewline
\providecommand{\tabularnewline}{\\}

%%%%%%%%%%%%%%%%%%%%%%%%%%%%%% Textclass specific LaTeX commands.
% this default might be overridden by plain title style
\newcommand\makebeamertitle{\frame{\maketitle}}%
% (ERT) argument for the TOC
\AtBeginDocument{%
  \let\origtableofcontents=\tableofcontents
  \def\tableofcontents{\@ifnextchar[{\origtableofcontents}{\gobbletableofcontents}}
  \def\gobbletableofcontents#1{\origtableofcontents}
}

%%%%%%%%%%%%%%%%%%%%%%%%%%%%%% User specified LaTeX commands.
\usetheme[secheader]{Boadilla}
\usecolortheme{seahorse}
\title[Natural transformations]{Properties of natural transformations}
\subtitle{With code examples in Scala}
\author{Sergei Winitzki}
\date{2020-05-30}
\institute[ABTB]{Academy by the Bay}
\setbeamertemplate{headline}{} % disable headline at top
\setbeamertemplate{navigation symbols}{} % disable navigation bar at bottom
\usepackage[all]{xy} % xypic
%\makeatletter
% Macros to assist LyX with XYpic when using scaling.
\newcommand{\xyScaleX}[1]{%
\makeatletter
\xydef@\xymatrixcolsep@{#1}
\makeatother
} % end of \xyScaleX
\makeatletter
\newcommand{\xyScaleY}[1]{%
\makeatletter
\xydef@\xymatrixrowsep@{#1}
\makeatother
} % end of \xyScaleY

% Double-stroked fonts to replace the non-working \mathbb{1}.
\usepackage{bbold}
\DeclareMathAlphabet{\bbnumcustom}{U}{BOONDOX-ds}{m}{n} % Use BOONDOX-ds or bbold.
\newcommand{\custombb}[1]{\bbnumcustom{#1}}
% The LyX document will define a macro \bbnum{#1} that calls \custombb{#1}.

\usepackage{relsize} % make math symbols larger or smaller
\usepackage{stmaryrd} % some extra symbols such as \fatsemi
% Note: using \forwardcompose inside a \text{} will cause a LaTeX error!
\newcommand{\forwardcompose}{\hspace{1.5pt}\ensuremath\mathsmaller{\fatsemi}\hspace{1.5pt}}


% Make underline green.
\definecolor{greenunder}{rgb}{0.1,0.6,0.2}
%\newcommand{\munderline}[1]{{\color{greenunder}\underline{{\color{black}#1}}\color{black}}}
\def\mathunderline#1#2{\color{#1}\underline{{\color{black}#2}}\color{black}}
% The LyX document will define a macro \gunderline{#1} that will use \mathunderline with the color `greenunder`.
%\def\gunderline#1{\mathunderline{greenunder}{#1}} % This is now defined by LyX itself with GUI support.

% Scala syntax highlighting. See https://tex.stackexchange.com/questions/202479/unable-to-define-scala-language-with-listings
%\usepackage[T1]{fontenc}
%\usepackage[utf8]{inputenc}
%\usepackage{beramono}
%\usepackage{listings}
% The listing settings are now supported by LyX in a separate section "Listings".
\usepackage{xcolor}

\definecolor{scalakeyword}{rgb}{0.16,0.07,0.5}
\definecolor{dkgreen}{rgb}{0,0.6,0}
\definecolor{gray}{rgb}{0.5,0.5,0.5}
\definecolor{mauve}{rgb}{0.58,0,0.82}
\definecolor{aqua}{rgb}{0.9,0.96,0.999}
\definecolor{scalatype}{rgb}{0.2,0.3,0.2}

\makeatother

\usepackage{listings}
\lstset{language=Scala,
morekeywords={{scala}},
otherkeywords={=,=>,<-,<\%,<:,>:,\#,@,:,[,],.,???},
keywordstyle={\color{scalakeyword}},
morekeywords={[2]{String,Short,Int,Long,Char,Boolean,Double,Float,BigDecimal,Seq,Map,Set,List,Option,Either,Future,Vector,Range,IndexedSeq,Try,true,false,None,Some,Left,Right,Nothing,Any,Array,Unit,Iterator,Stream}},
keywordstyle={[2]{\color{scalatype}}},
frame=tb,
aboveskip={1.5mm},
belowskip={0.5mm},
showstringspaces=false,
columns=fullflexible,
keepspaces=true,
basicstyle={\smaller\ttfamily},
extendedchars=true,
numbers=none,
numberstyle={\tiny\color{gray}},
commentstyle={\color{dkgreen}},
stringstyle={\color{mauve}},
frame=single,
framerule={0.0mm},
breaklines=true,
breakatwhitespace=true,
tabsize=3,
framexleftmargin={0.5mm},
framexrightmargin={0.5mm},
xleftmargin={1.5mm},
xrightmargin={1.5mm},
framextopmargin={0.5mm},
framexbottommargin={0.5mm},
fillcolor={\color{aqua}},
rulecolor={\color{aqua}},
rulesepcolor={\color{aqua}},
backgroundcolor={\color{aqua}},
mathescape=false,
extendedchars=true}
\renewcommand{\lstlistingname}{Listing}

\begin{document}
\global\long\def\gunderline#1{\mathunderline{greenunder}{#1}}%
\global\long\def\bef{\forwardcompose}%
\global\long\def\bbnum#1{\custombb{#1}}%
\frame{\titlepage}
\begin{frame}{Refactoring code by permuting the order of operations}
\begin{itemize}
\item \vspace{-0.15cm}Expected properties of refactored code: \\
~
\end{itemize}
First extract user information, then convert stream to list; or first
convert to list, then extract user information:

\texttt{\textcolor{blue}{\footnotesize{}db.getRows.toList.map(getUserInfo)}}
gives the same result as\\
\texttt{\textcolor{blue}{\footnotesize{}db.getRows.map(getUserInfo).toList}}~\\
\texttt{\textcolor{blue}{\footnotesize{}~}}{\footnotesize\par}

First extract user information, then exclude invalid rows; or first
exclude invalid rows, then extract user information:

\texttt{\textcolor{blue}{\footnotesize{}db.getRows.map(getUserInfo).filter(isValid)}}
gives the same result as \\
\texttt{\textcolor{blue}{\footnotesize{}db.getRows.filter(getUserInfo
andThen isValid).map(getUserInfo)}}~\\
\texttt{\textcolor{blue}{\footnotesize{}~}}{\footnotesize\par}
\begin{itemize}
\item These refactorings are guaranteed to be correct...
\begin{itemize}
\item ... because \texttt{\textcolor{blue}{\footnotesize{}\_.toList}} is
a natural transformation \texttt{\textcolor{blue}{\footnotesize{}Stream{[}A{]}
=> List{[}A{]}}} 
\end{itemize}
\end{itemize}
\end{frame}

\begin{frame}{Refactored code: equations}

Introduce short syntax to write those properties as equations:
\begin{center}
\begin{tabular}{|c|c|}
\hline 
\texttt{\textcolor{blue}{\small{}def toList{[}A{]}: Stream{[}A{]}
=> List{[}A{]}}} & {\small{}$\text{toList}^{A}:\text{Str}^{A}\rightarrow\text{List}^{A}$}\tabularnewline
\hline 
\texttt{\textcolor{blue}{\small{}val f: A => B}} & {\small{}$f^{:A\rightarrow B}$}\tabularnewline
\hline 
\texttt{\textcolor{blue}{\small{}\_.map(f)}}{\small{} with type }\texttt{\textcolor{blue}{\small{}List{[}A{]}
=> List{[}B{]}}} & {\small{}$f^{\uparrow\text{List}}$}\tabularnewline
\hline 
\texttt{\textcolor{blue}{\small{}\_.toList.map(f)}} & {\small{}$\text{toList}\bef f^{\uparrow\text{List}}$}\tabularnewline
\hline 
\texttt{\textcolor{blue}{\small{}f andThen g}} & {\small{}$f\bef g$}\tabularnewline
\hline 
\texttt{\textcolor{blue}{\small{}\_.map(f).map(g) == \_.map(f andThen
g)}} & {\small{}$f^{\uparrow\text{List}}\bef g^{\uparrow\text{List}}=(f\bef g)^{\uparrow\text{List}}$}\tabularnewline
\hline 
\end{tabular}
\par\end{center}

The ``short syntax'' is equivalent to Scala code
\end{frame}

\begin{frame}{Refactored code: equations}

\vspace{-0.2cm}Rewrite the previous examples as equations and type
diagrams:

\vspace{0.2cm}\lstinline!def toList[A]: Stream[A] => List[A]! written
as {\small{}$\text{toList}^{A}:\text{Str}^{A}\rightarrow\text{List}^{A}$}{\small\par}

\vspace{-0.3cm}%
\begin{minipage}[c][1\totalheight][t]{0.26\columnwidth}%
\[
\xymatrix{\text{Str}^{A}\ar[r]\sp(0.55){\text{toList}^{A}}\ar[d]\sp(0.4){f^{\uparrow\text{Str}}} & \text{List}^{A}\ar[d]\sp(0.4){f^{\uparrow\text{List}}}\\
\xyScaleY{1.6pc}\xyScaleX{3.0pc}\text{Str}^{B}\ar[r]\sp(0.55){\text{toList}^{B}} & \text{List}^{B}
}
\]
%
\end{minipage}\hfill{}%
\begin{minipage}[c][1\totalheight][t]{0.68\columnwidth}%
\begin{center}
~\\
~\\
\lstinline!_.toList.map(f) == _.map(f).toList!{\small{}
\[
\text{toList}^{A}\bef f^{\uparrow\text{List}}=f^{\uparrow\text{Str}}\bef\text{toList}^{B}
\]
}
\par\end{center}%
\end{minipage}

\vspace{0.2cm}\lstinline!def filt[A]: (A => Boolean) => Stream[A] => Stream[A]!

\vspace{-0.3cm}%
\begin{minipage}[c][1\totalheight][t]{0.26\columnwidth}%
\[
\xymatrix{\text{Str}^{A}\ar[r]\sp(0.55){\text{filt}^{A}(f\bef p)}\ar[d]\sp(0.4){f^{\uparrow\text{Str}}} & \text{Str}^{A}\ar[d]\sp(0.4){f^{\uparrow\text{Str}}}\\
\xyScaleY{1.6pc}\xyScaleX{3.0pc}\text{Str}^{B}\ar[r]\sp(0.55){\text{filt}^{B}(p)} & \text{Str}^{B}
}
\]
%
\end{minipage}\hfill{}%
\begin{minipage}[c][1\totalheight][t]{0.68\columnwidth}%
\begin{center}
{\small{}
\begin{align*}
 & \text{filt}^{A}:(A\rightarrow\bbnum 2)\rightarrow\text{Str}^{A}\rightarrow\text{Str}^{A}\\
 & f^{\uparrow\text{Str}}\bef\text{filt}^{B}(p)=\text{filt}^{A}(f\bef p)\bef f^{\uparrow\text{Str}}
\end{align*}
}
\par\end{center}%
\end{minipage}
\begin{itemize}
\item \vspace{0.2cm}A transformation before \lstinline!map! equals a transformation
after \lstinline!map!
\item This is called a \textbf{naturality law}
\item We expect it to hold if the code works the same way for all types
\begin{itemize}
\item The naturality law is a mathematical expression of the programmer's
intuition about code ``working the same way for all types''
\end{itemize}
\end{itemize}
\end{frame}

\begin{frame}{Naturality laws: equations}

\vspace{-0.1cm}\textbf{Naturality law} for a function $t$ is an
equation involving an arbitrary function $f$ that permutes the order
of application of $t$ and of a lifted $f$

\vspace{-0.4cm}%
\begin{minipage}[c][1\totalheight][t]{0.26\columnwidth}%
\[
\xymatrix{\text{List}^{A}\ar[r]\sp(0.55){\text{headOpt}^{A}}\ar[d]\sp(0.4){f^{\uparrow\text{List}}} & \text{Opt}^{A}\ar[d]\sb(0.4){f^{\uparrow\text{Opt}}}\\
\xyScaleY{1.6pc}\xyScaleX{3.0pc}\text{List}^{B}\ar[r]\sp(0.55){\text{headOpt}^{B}} & \text{Opt}^{B}
}
\]
%
\end{minipage}\hfill{}%
\begin{minipage}[c][1\totalheight][b]{0.68\columnwidth}%
\begin{center}
\lstinline!list.map(f).headOption == list.headOption.map(f)!
\[
(f^{:A\rightarrow B})^{\uparrow\text{List}}\bef\text{headOpt}=\text{headOpt}\bef(f^{:A\rightarrow B})^{\uparrow\text{Opt}}
\]
\par\end{center}%
\end{minipage}
\begin{itemize}
\item Lifting $f$ before $t$ equals to lifting $f$ after $t$
\item Intuition: $t$ rearranges data in a collection, not looking at values
\end{itemize}
Further examples: 
\begin{itemize}
\item Reversing a list; $\text{reverse}^{A}:\text{List}^{A}\rightarrow\text{List}^{A}$
\end{itemize}
\begin{center}
\lstinline!list.map(f).reverse == list.reverse.map(f)!{\footnotesize{}
\[
(f^{:A\rightarrow B})^{\uparrow\text{List}}\bef\text{reverse}^{B}=\text{reverse}^{A}\bef(f^{:A\rightarrow B})^{\uparrow\text{List}}
\]
}{\footnotesize\par}
\par\end{center}
\begin{itemize}
\item The \lstinline!pure! method, \lstinline!pure[A]: A => L[A]!. Notation:
$\text{pu}_{L}:A\rightarrow L^{A}$
\end{itemize}
\begin{center}
\lstinline!pure(x).map(f) == pure(f(x))!{\footnotesize{}
\[
\text{pu}^{A}\bef(f^{:A\rightarrow B})^{\uparrow L}=f\bef\text{pu}^{B}
\]
}{\footnotesize\par}
\par\end{center}

\end{frame}

\begin{frame}{Natural transformations and their laws}

A \textbf{natural transformation} is a function $t$ with type signature
$F^{A}\rightarrow G^{A}$ that satisfies the naturality law $f^{\uparrow F}\bef t=t\bef f^{\uparrow G}$.
Notation $t:F\leadsto G$

Mnemonic rule: if $t:F\leadsto G$ then the lifting to $F$ is on
the left, the lifting to $G$ is on the right
\begin{itemize}
\item Many standard methods have the form of a natural transformation
\begin{itemize}
\item Examples: \lstinline!headOption!, \lstinline!lastOption!, \lstinline!reverse!,
\lstinline!swap!, \lstinline!map!, \lstinline!flatMap!, \lstinline!pure!
\end{itemize}
\item If there are several type parameters, use one at a time:
\begin{itemize}
\item For \lstinline!flatMap!, denote $\text{flm}:\left(A\rightarrow M^{B}\right)\rightarrow M^{A}\rightarrow M^{B}$,
fix $A$
\begin{itemize}
\item $\text{flm}:F^{B}\rightarrow G^{B}$ where $F^{B}\triangleq A\rightarrow M^{B}$
and $G^{B}\triangleq M^{A}\rightarrow M^{B}$
\end{itemize}
\item The naturality law $f^{\uparrow F}\bef\text{flm}=\text{flm}\bef f^{\uparrow G}$
then gives the equation
\[
\text{flm}\,(p^{:A\rightarrow M^{B}}\bef f^{\uparrow M})=\text{flm}\,(p^{:A\rightarrow M^{B}})\bef f^{\uparrow M}
\]
if we write out the code for $f^{\uparrow F}$ and $f^{\uparrow G}$:
\[
f^{\uparrow F}=p^{:A\rightarrow M^{B}}\rightarrow p\bef f^{\uparrow M}\quad,\quad\quad f^{\uparrow G}=q^{:M^{A}\rightarrow M^{B}}\rightarrow q\bef f^{\uparrow M}
\]
\end{itemize}
\end{itemize}
\end{frame}

\begin{frame}{More practical uses of natural transformations I}

Recognize natural transformations in code and refactor

\vspace{0.3cm}\texttt{\textcolor{blue}{\footnotesize{}def ensureName(name:~Option{[}String{]},
id:~Long):~Option{[}(String, Long){]} =}}{\footnotesize\par}

\texttt{\textcolor{blue}{\footnotesize{}~ ~ ~ name.map((\_, id))}}{\footnotesize\par}
\begin{itemize}
\item Recognize that the code works the same way for all types
\item Introduce type parameters \texttt{\textcolor{blue}{\footnotesize{}A}}
and \texttt{\textcolor{blue}{\footnotesize{}B}} instead of \texttt{\textcolor{blue}{\footnotesize{}String}}
and \texttt{\textcolor{blue}{\footnotesize{}Long}} 
\item The refactored code is a natural transformation:
\end{itemize}
\texttt{\textcolor{blue}{\footnotesize{}def toOptionPair{[}A, B{]}(x:~Option{[}A{]},
b:~B):~Option{[}(A, B){]} = }}{\footnotesize\par}

\texttt{\textcolor{blue}{\footnotesize{}~ ~ ~ x.map((\_, b))}}{\footnotesize\par}

The type signature is of the form \texttt{\textcolor{blue}{\footnotesize{}F{[}A{]}
=> G{[}A{]}}} if we define

\texttt{\textcolor{blue}{\footnotesize{}type F{[}A{]} = (Option{[}A{]},
B{]})}}{\footnotesize\par}

\texttt{\textcolor{blue}{\footnotesize{}type G{[}A{]} = Option{[}(A,
B){]}}}{\footnotesize\par}

and consider \texttt{\textcolor{blue}{\footnotesize{}B}} as a fixed
type

Alternatively, consider \texttt{\textcolor{blue}{\footnotesize{}A}}
as a fixed type and obtain a natural transformation \texttt{\textcolor{blue}{\footnotesize{}K{[}B{]}
=> L{[}B{]}}} with suitable definitions of \texttt{\textcolor{blue}{\footnotesize{}K{[}B{]}}}
and \texttt{\textcolor{blue}{\footnotesize{}L{[}B{]}}} 
\begin{itemize}
\item The naturality law can be verified directly
\begin{itemize}
\item But it also follows from the parametricity theorem
\end{itemize}
\end{itemize}
\end{frame}

\begin{frame}{More practical uses of natural transformations II}

Building up natural transformations from parts

\vspace{0.3cm}\texttt{\textcolor{blue}{\footnotesize{}def toOptionList{[}A,
B{]}:~List{[}(Option{[}A{]}, B){]} => List{[}Option{[}(A, B){]}{]}
=}}{\footnotesize\par}

\texttt{\textcolor{blue}{\footnotesize{}~ ~ ~ \_.map \{ case (x,
b) => x.map((\_, id)) \}}}{\footnotesize\par}
\begin{itemize}
\item If we have a functor $F$ and a natural transformation $G^{A}\rightarrow H^{A}$,
we can implement a natural transformation $F^{G^{A}}\rightarrow F^{H^{A}}$
\item In this example, the notation is $F=\text{List}$, $G^{A}=(\bbnum 1+A)\times B$,
and $H^{A}=\bbnum 1+A\times B$
\begin{itemize}
\item The type notation such as $(\bbnum 1+A)\times B$ helps recognize
type equivalences by using the rules of ordinary polynomial algebra:
\[
(\bbnum 1+A)\times B\cong\bbnum 1\times B+A\times B\cong B+A\times B
\]
\end{itemize}
\item Another example: \texttt{\textcolor{blue}{\footnotesize{}List{[}(Try{[}A{]},
B){]} => List{[}Try{[}(A, B){]}{]}}} with the same code
\item Denote \texttt{\textcolor{blue}{\footnotesize{}Try{[}A{]}}} by $E+A$
where $E$ denotes the type of the exception
\[
\text{List}^{\left(E+A\right)\times B}\rightarrow\text{List}^{E+A\times B}
\]
\end{itemize}
\end{frame}

\begin{frame}{More practical uses of natural transformations III}

Using a constant functor (``phantom type parameter'')

\vspace{0.3cm}\texttt{\textcolor{blue}{\footnotesize{}def length{[}A{]}:~List{[}A{]}
=> Int = \{ \_.length \}}}{\footnotesize\par}
\begin{itemize}
\item The type signature is of the form \texttt{\textcolor{blue}{\footnotesize{}F{[}A{]}
=> G{[}A{]}}} or $F^{A}\rightarrow G^{A}$ if we define $F=\text{List}$
and $G^{A}=\text{Int}$, so that $G^{A}$ is a constant functor
\item The naturality law gives $f^{\uparrow F}\bef\text{length}=\text{length}\bef f^{\uparrow G}$,
but $F^{\uparrow G}=\text{id}$, so $f^{\uparrow F}\bef\text{length}=\text{length}$
for any $f^{:A\rightarrow B}$
\item We can choose $f(x)=c$ with any constant $c$ 
\begin{itemize}
\item The length of a list does not depend on the values stored in the list
\end{itemize}
\end{itemize}
\end{frame}

\begin{frame}{Reasoning with naturality: Simplifying the \lstinline!pure! method}

\vspace{-0.2cm}The naturality law of \lstinline!pure! for a functor
$L$:

\vspace{-0.4cm}%
\begin{minipage}[c][1\totalheight][t]{0.26\columnwidth}%
\[
\xymatrix{A\ar[r]\sp(0.55){\text{pu}_{L}}\ar[d]\sp(0.4){f} & L^{A}\ar[d]\sp(0.4){f^{\uparrow L}}\\
\xyScaleY{1.6pc}\xyScaleX{3.0pc}B\ar[r]\sp(0.55){\text{pu}_{L}} & L^{B}
}
\]
%
\end{minipage}\hfill{}%
\begin{minipage}[c][1\totalheight][b]{0.68\columnwidth}%
\begin{center}
\lstinline!pure(a).map(f) == pure(f(a))!
\[
\text{pu}_{L}\bef f^{\uparrow L}=f\bef\text{pu}_{L}
\]
\par\end{center}%
\end{minipage}

\vspace{0.2cm}Fix a value $b^{:B}$ and set $A=\bbnum 1$ and $f\triangleq1\rightarrow b$
in the naturality law:

\vspace{-0.4cm}%
\begin{minipage}[c][1\totalheight][t]{0.26\columnwidth}%
\[
\xymatrix{\bbnum 1\ar[r]\sp(0.55){\text{pu}_{L}}\ar[d]\sp(0.4){1\rightarrow b} & L^{\bbnum 1}\ar[d]\sp(0.4){(1\rightarrow b)^{\uparrow L}}\\
\xyScaleY{1.6pc}\xyScaleX{3.0pc}B\ar[r]\sp(0.55){\text{pu}_{L}} & L^{B}
}
\]
%
\end{minipage}\hfill{}%
\begin{minipage}[c][1\totalheight][b]{0.68\columnwidth}%
\begin{center}
\lstinline!pure(()).map(_ => b) == pure(b)!
\[
\text{pu}_{L}\bef(1\rightarrow b)^{\uparrow L}=(1\rightarrow b)\bef\text{pu}_{L}
\]
\par\end{center}%
\end{minipage}

\vspace{0.2cm}We have expressed \lstinline!pure(b)! via a constant
value \lstinline!pure(())! of type \lstinline!L[Unit]!

The resulting function \lstinline!pure! will automatically satisfy
the naturality law!

The naturality law of \lstinline!pure! makes it \emph{equivalent}
to a ``wrapped unit'' value

This simplifies the definition of a \lstinline!Pointed! typeclass:

\lstinline!abstract class Pointed[L[_]: Functor] \{ def wu: L[Unit] \}!

Examples: for \lstinline!Option!, \lstinline!wu = Some(())!. For
\lstinline!List!, \lstinline!wu = List(())!
\end{frame}

\begin{frame}{Reasoning with naturality: \lstinline!flatMap! and \lstinline!flatten!}

Use the curried type signature for \lstinline!flatMap! for a monad
$M$:

\vspace{0.3cm}\texttt{\textcolor{blue}{\footnotesize{}def flatMap{[}A,
B{]}:~(A => M{[}B{]})=> M{[}A{]} => M{[}B{]}}} 
\[
\text{flm}^{A,B}:(A\rightarrow M^{B})\rightarrow M^{A}\rightarrow M^{B}
\]

The naturality law with respect to the type parameter \texttt{\textcolor{blue}{\footnotesize{}A}}:

\vspace{-0.4cm}%
\begin{minipage}[c][1\totalheight][t]{0.26\columnwidth}%
\[
\xymatrix{M^{A}\ar[rd]\sp(0.55){\text{flm}\,(f\bef g)}\ar[d]\sb(0.4){f^{\uparrow M}}\\
\xyScaleY{1.6pc}\xyScaleX{4.0pc}M^{B}\ar[r]\sp(0.45){\text{flm}\,(g)} & M^{C}
}
\]
%
\end{minipage}\hfill{}%
\begin{minipage}[c][1\totalheight][b]{0.68\columnwidth}%
\begin{center}
\lstinline!_.flatMap(f andThen g) == _.map(f).flatMap(g)!
\[
\text{flm}\,(f^{:A\rightarrow B}\bef g^{:B\rightarrow M^{C}})=f^{\uparrow M}\bef\text{flm}\,(g)\quad.
\]
\par\end{center}%
\end{minipage}

\vspace{0.2cm}Express \lstinline!flatMap! through \lstinline!flatten!:

\lstinline!_.flatMap(f) == _.map(f).flatten!
\[
\text{flm}\,(g)=g^{\uparrow M}\bef\text{ftn}
\]
Express \lstinline!flatten! through \lstinline!flatMap!:
\[
\text{ftn}=\text{flm}\,(\text{id}^{:M^{A}\rightarrow M^{A}})
\]
The function \lstinline!flatten! is equivalent to \lstinline!flatMap!
with naturality law
\end{frame}

\begin{frame}{The covariant Yoneda identity}

We have shown that the set of all natural transformations $A\rightarrow L^{A}$
is equivalent to the set of all values $L^{\bbnum 1}$

This property can be generalized to any type $Z$ instead of the unit
type ($\bbnum 1$):

The set of all natural transformations $\left(Z\rightarrow A\right)\rightarrow L^{A}$
is equivalent to the set of all values $L^{Z}$, where $Z$ is a fixed
type

To indicate that $Z$ is fixed by $A$ is varying within the natural
transformation, use a type signature with the universal quantifier:
\begin{align*}
\big(\forall A.\,A & \rightarrow L^{A}\big)\cong L^{\bbnum 1}\\
\big(\forall A.\,\left(Z\rightarrow A\right) & \rightarrow L^{A}\big)\cong L^{Z}\quad\quad\text{-- the covariant Yoneda identity}
\end{align*}

To prove:
\begin{enumerate}
\item Implement the isomorphism, $p:\big(\forall A.\,\left(Z\rightarrow A\right)\rightarrow L^{A}\big)\rightarrow L^{Z}$
and $q:L^{Z}\rightarrow\forall A.\,\left(Z\rightarrow A\right)\rightarrow L^{A}$
\item Show that $p\bef q=\text{id}$ and $q\bef p=\text{id}$
\end{enumerate}
\end{frame}

\begin{frame}{Reasoning with naturality laws}

Naturality laws are often used in derivations of various typeclass
laws

Within the 11 existing chapters of my upcoming free book, ``\emph{The
Science of Functional Programming}'' (\texttt{\small{}\href{https://github.com/winitzki/sofp}{https://github.com/winitzki/sofp}}),
naturality laws are used at least 31 times in about 100 derivations
\begin{itemize}
\item Examples of such derivations:
\begin{itemize}
\item Composition of two co-pointed functors is again co-pointed
\begin{itemize}
\item A functor $F$ is co-pointed if there exists a natural transformation
$\text{ex}:\forall A.\,F^{A}\rightarrow A$
\end{itemize}
\item The product of two monads is again a monad
\item The product of two monad transformers is again a monad transformer
\end{itemize}
\end{itemize}
The most useful derivation technique is rewriting equations
\end{frame}

\begin{frame}{Example: properties of horizontal and vertical composition}

Bartosz Milewski's book ``Category theory for programmers'', Chapter
10, defines the horizontal and the vertical composition of natural
transformations

The horizontal composition of $\alpha:F^{A}\rightarrow G^{A}$ and
$\beta:G^{A}\rightarrow H^{A}$ is the ordinary function composition
$\left(\alpha\bef\beta\right):F^{A}\rightarrow H^{A}$

The vertical composition of $\alpha:F^{A}\rightarrow G^{A}$ and $\alpha^{\prime}:F^{\prime A}\rightarrow G^{\prime A}$
is $\left(\alpha\star\alpha^{\prime}\right):F^{F^{\prime A}}\rightarrow G^{G^{\prime A}}$
\begin{itemize}
\item Both compositions again give natural transformations
\end{itemize}
If we have four natural transformations $\alpha$, $\beta$, $\alpha^{\prime}$,
$\beta^{\prime}$ with type signatures

\begin{align*}
\alpha & :F^{A}\rightarrow G^{A}\quad,\quad\beta:G^{A}\rightarrow H^{A}\quad,\\
\alpha^{\prime} & :F^{\prime A}\rightarrow G^{\prime A}\quad,\quad\beta^{\prime}:G^{\prime A}\rightarrow H^{\prime A}\quad,
\end{align*}
we can write the distributive law,
\[
\left(\alpha\bef\beta\right)\star(\alpha^{\prime}\bef\beta^{\prime})=(\alpha\star\alpha^{\prime})\bef(\beta\star\beta^{\prime})
\]
To prove that these properties hold, write out the naturality laws
\end{frame}

\begin{frame}{Other constructions of natural transformations}

Natural transformations can be combined in several other ways

Given two natural transformations \texttt{\textcolor{blue}{\footnotesize{}a{[}A{]}: F{[}A{]}
=> G{[}A{]}}} and \texttt{\textcolor{blue}{\footnotesize{}b{[}A{]}: K{[}A{]}
=> L{[}A{]}}}:
\begin{itemize}
\item Pair product: \texttt{\textcolor{blue}{\footnotesize{}((F{[}A{]},
K{[}A{]})) => (G{[}A{]}, L{[}A{]})}} 
\item Pair co-product: \texttt{\textcolor{blue}{\footnotesize{}Either{[}F{[}A{]},
K{[}A{]}{]} => Either{[}G{[}A{]}, L{[}A{]}{]}}} 
\item Pair exponential: \texttt{\textcolor{blue}{\footnotesize{}(F{[}A{]}
=> K{[}A{]}) => (G{[}A{]} => L{[}A{]})}} where \texttt{\textcolor{blue}{\footnotesize{}F{[}A{]}}}
and \texttt{\textcolor{blue}{\footnotesize{}G{[}A{]}}} must be contrafunctors
\end{itemize}
Also, the identity function \texttt{\textcolor{blue}{\footnotesize{}identity{[}A{]}: A
=> A}} and the constant unit function of type \texttt{\textcolor{blue}{\footnotesize{}A
=> Unit}} are natural transformations

It follows that any purely functional combination of natural transformations
is again a natural transformation; no need to verify the naturality
law in each case
\end{frame}

\begin{frame}{Summary of the type notation}

The short type notation helps in symbolic reasoning about types
\noindent \begin{center}
\begin{tabular}{|c|c|c|}
\hline 
\textbf{\small{}Description} & \textbf{\small{}Scala examples} & \textbf{\small{}Notation}\tabularnewline
\hline 
\hline 
{\footnotesize{}Typed value} & {\footnotesize{}}\lstinline!x: Int! & {\footnotesize{}$x^{:\text{Int}}$ or $x:\text{Int}$}\tabularnewline
\hline 
{\footnotesize{}Unit type} & {\footnotesize{}}\lstinline!Unit!{\footnotesize{}, }\lstinline!Nil!{\footnotesize{},
}\lstinline!None! & {\footnotesize{}$\bbnum 1$}\tabularnewline
\hline 
{\footnotesize{}Type parameter} & {\footnotesize{}}\lstinline!A! & {\footnotesize{}$A$}\tabularnewline
\hline 
{\footnotesize{}Product type} & {\footnotesize{}}\lstinline!(A, B)!{\footnotesize{} or }\lstinline!case class P(x: A, y: B)! & {\footnotesize{}$A\times B$}\tabularnewline
\hline 
{\footnotesize{}Co-product type} & {\footnotesize{}}\lstinline!Either[A, B]! & {\footnotesize{}$A+B$}\tabularnewline
\hline 
{\footnotesize{}Function type} & {\footnotesize{}}\lstinline!A => B! & {\footnotesize{}$A\rightarrow B$}\tabularnewline
\hline 
{\footnotesize{}Type constructor} & {\footnotesize{}}\lstinline!List[A]! & {\footnotesize{}$\text{List}^{A}$}\tabularnewline
\hline 
{\footnotesize{}Universal quantifier} & {\footnotesize{}}\lstinline!trait P \{ def f[A]: Q[A] \}! & {\footnotesize{}$P\triangleq\forall A.\,Q^{A}$}\tabularnewline
\hline 
{\footnotesize{}Existential quantifier} & {\footnotesize{}}%
\begin{minipage}[t]{0.43\paperwidth}%
{\footnotesize{}}\lstinline!sealed trait P[A]!{\footnotesize\par}

{\footnotesize{}}\lstinline!case class Q[A, B]() extends P[A]!{\footnotesize{}\vspace{0.2\baselineskip}
}{\footnotesize\par}%
\end{minipage} & {\footnotesize{}$P^{A}\triangleq\exists B.\,Q^{A,B}$}\tabularnewline
\hline 
\end{tabular}
\par\end{center}

Example: Scala code \lstinline!def flm(f: A => Option[B]): Option[A] => Option[B]!
is denoted by $\text{flm}:(A\rightarrow\bbnum 1+B)\rightarrow\bbnum 1+A\rightarrow\bbnum 1+B$
\end{frame}

\begin{frame}{Summary of the code notation}

The short code notation helps in symbolic reasoning about code
\noindent \begin{center}
\begin{tabular}{|c|c|}
\hline 
\textbf{\small{}Scala examples} & \textbf{\small{}Notation}\tabularnewline
\hline 
\hline 
{\small{}}\lstinline!()!{\small{} or }\lstinline!true!{\small{}
or }\lstinline!"abc"!{\small{} or }\lstinline!123! & {\small{}$1$, $\text{true}$, $\text{"abc"}$, $123$}\tabularnewline
\hline 
{\small{}}\lstinline!def f[A](x: A) = ...! & {\small{}$f^{A}(x^{:A})\triangleq...$}\tabularnewline
\hline 
{\small{}}\lstinline!\{ (x: A) => expr \}! & {\small{}$x^{:A}\rightarrow\text{expr}$}\tabularnewline
\hline 
{\small{}}\lstinline!f(x)!{\small{} or }\lstinline!x.pipe(f)!{\small{}
(Scala 2.13)} & {\small{}$f(x)$ or $x\triangleright f$}\tabularnewline
\hline 
{\small{}}\lstinline!val p: (A, B) = (a, b)! & {\small{}$p^{:A\times B}\triangleq a\times b$}\tabularnewline
\hline 
{\small{}}\lstinline!\{case (a, b) => expr\}!{\small{} or }\lstinline!p._1!{\small{}
or }\lstinline!p._2! & {\small{}$a\times b\rightarrow\text{expr}$ ~or~ $p\triangleright\pi_{1}$~or~
$p\triangleright\pi_{2}$}\tabularnewline
\hline 
{\small{}}\lstinline!Left[A, B](x)!{\small{} or }\lstinline!Right[A, B](y)! & {\small{}$x^{:A}+\bbnum 0^{:B}$ or $\bbnum 0^{:A}+y^{:B}$}\tabularnewline
\hline 
{\small{}\hspace*{-0.013\linewidth}}%
\begin{minipage}[c][1\totalheight][b]{0.5\columnwidth}%
{\small{}\vspace{0.14\baselineskip}
}{\footnotesize{}}\lstinline!val q: C = (p: Either[A, B]) match \{!{\footnotesize\par}

{\footnotesize{}~ ~}\lstinline!case Left(x)   => f(x)!{\footnotesize\par}

{\footnotesize{}~ ~}\lstinline!case Right(y)  => g(y)!{\footnotesize\par}

{\footnotesize{}}\lstinline!\}!{\small{}\vspace{0.2\baselineskip}
}{\small\par}%
\end{minipage}{\small{} \hspace*{-0.009\linewidth}} & {\small{}$q^{:C}\triangleq p^{:A+B}\triangleright\begin{array}{|c||c|}
 & C\\
\hline A & x^{:A}\rightarrow f(x)\\
B & y^{:B}\rightarrow g(y)
\end{array}$}\tabularnewline
\hline 
{\small{}}\lstinline!def f(x) = \{ ... f(y) ... \}! & {\small{}}%
\begin{minipage}[t]{0.3\columnwidth}%
\begin{center}
{\small{}\vspace{-0.64\baselineskip}
$f(x)\triangleq...~\overline{f}(y)~...$\vspace{0.15\baselineskip}
}
\par\end{center}%
\end{minipage}\tabularnewline
\hline 
{\small{}}\lstinline!f andThen g!{\small{} and }\lstinline!(f andThen g)(x)! & {\small{}$f\bef g$ and $x\triangleright f\bef g$~or~ $x\triangleright f\triangleright g$}\tabularnewline
\hline 
{\small{}}\lstinline!p.map(f).map(g)! & {\small{}$p\triangleright f^{\uparrow F}\triangleright g^{\uparrow F}$~or~
$p\triangleright f^{\uparrow F}\bef g^{\uparrow F}$}\tabularnewline
\hline 
\end{tabular}
\par\end{center}

\end{frame}

\begin{frame}{Summary}
\begin{itemize}
\item Naturality laws can be used for guaranteed correct refactoring
\begin{itemize}
\item Naturality laws allow us to reduce the number of type parameters
\end{itemize}
\item Full details and proofs are in the free upcoming book
\begin{itemize}
\item Draft of the book: \texttt{\href{https://github.com/winitzki/sofp}{https://github.com/winitzki/sofp}}
\end{itemize}
\end{itemize}
\end{frame}

\end{document}
