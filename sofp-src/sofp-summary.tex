
\chapter{Sample problems}
\begin{enumerate}
\item Compute the smallest integer expressible as a sum of two cubed integers
in more than one way.
\item Read a text file, split it by spaces into words, and print the word
counts, sorted by decreasing count.
\item FPIS exercise 2.2: Check whether a sequence \lstinline!Seq[A]! is
sorted according to a given ordering function of type \lstinline!(A, A) => Boolean!.
\item FPIS exercise 3.24: Implement a function \lstinline!hasSubsequence!
that checks whether a \lstinline!List! contains another \lstinline!List!
as a subsequence. For instance, \lstinline!List(1,2,3,4)! would have
\lstinline!List(1,2)!, \lstinline!List(2,3)!, and \lstinline!List(4)!
as subsequences, among others. (Dynamic programming?)
\item (Bird, de Moor page 20) Derive the following identity between functions
$F^{A}\rightarrow F^{A}$, for any filterable functor $F$ and any
predicate $p^{:A\rightarrow\bbnum 2}$: 
\[
\text{filt}_{F}(p)=(\Delta\bef\text{id}\boxtimes p)^{\uparrow F}\bef\text{filt}_{F}(\pi_{2})\bef\pi_{1}^{\uparrow F}\quad.
\]
\item Define a monoid of partial functions with fixed types $P\rightarrow Q$.
\begin{lstlisting}
final case class PFM[P, Q](pf: PartialFunction[P, Q])
// After defining a monoid instance, the following code must work:
val p1 = PFM[Option[Int], String] { case Some(3) => "three" }
val p2 = PFM[Option[Int], String] {
  case Some(20)   => "twenty"
  case None       => "empty"
}
p1 |+| p2 // Must be the same as the concatenation of all `case` clauses.
\end{lstlisting}
\item Consider a typeclass called \textsf{``}\lstinline!Splittable!\textsf{''} for functors
$F^{\bullet}$ that have an additional method 
\[
\text{split}^{A,B}:F^{A+B}\rightarrow F^{A}+B
\]
with the non-degeneracy law for functions $F^{A}\rightarrow F^{A}$,
\[
(x^{:A}\rightarrow x+\bbnum 0^{:B})^{\uparrow F}\bef\text{split}=y^{:F^{A}}\rightarrow y+\bbnum 0^{:B}
\]
and the special associativity law for functions $F^{A+B+C}\rightarrow F^{A}+B+C$,
\[
\text{split}^{A+B,C}\bef\begin{array}{|c||cc|}
 & F^{A}+B & C\\
\hline F^{A+B} & \text{split}^{A,B} & \bbnum 0\\
C & \bbnum 0 & \text{id}
\end{array}=\text{split}^{A,B+C}\quad.
\]
Show that all polynomial functors $F^{\bullet}$ belong to this typeclass.
Show that exponential functors such as $F^{A}\triangleq Z\rightarrow A$
do not. 
\item Given two fixed types $P$, $Q$ that are not known to be the same,
consider the contrafunctors $F^{A}\triangleq\left(\left(A\rightarrow P\right)\rightarrow P\right)\rightarrow Q$
and $G^{A}\triangleq A\rightarrow Q$. Show that there exist natural
transformations $F^{\bullet}\leadsto G^{\bullet}$ and $G^{\bullet}\leadsto F^{\bullet}$.
Show that these transformations are not isomorphisms.
\item Given two fixed types $P$, $Q$ that are not known to be the same,
show that the functor $L^{A}\triangleq\left(\left(\left(A\rightarrow P\right)\rightarrow Q\right)\rightarrow Q\right)\rightarrow P$
is a semimonad but not a full monad. (When $P\cong Q$, the functor
$L$ is also not a full monad because $L$ is equivalent to a composition
of the continuation monad with itself. See Exercise~\ref{subsec:Exercise-monad-composition-mm}.)
\item When $M$ is a given semimonad, show that $M\circ M\circ...\circ M$
(with finitely many $M$) is also a semimonad.
\end{enumerate}

\subsection{Problems\index{problems}}

The author of this book does not know how to resolve these questions.

\subsubsection{Problem \label{par:Problem-monads}\ref{par:Problem-monads}}

Are there any polynomial monads not obtained by a chain of constructions
shown in Section~\ref{subsec:Constructions-of-polynomial-monads}?

\subsubsection{Problem \label{par:Problem-monads-1}\ref{par:Problem-monads-1}}

Do all polynomial functors of the form $P_{n}^{A}\triangleq\bbnum 1+\underbrace{A\times A\times...\times A}_{n\text{ times, }n\ge2}$
fail to be monads? An example is the functor $P_{2}^{A}\triangleq\bbnum 1+A\times A$,
which is not a monad because the monad laws are not satisfied by any
implementations of \lstinline!pure! and \lstinline!flatMap! methods
for that functor.\footnote{See discussion here: \texttt{\href{https://stackoverflow.com/questions/49742377}{https://stackoverflow.com/questions/49742377}}}

\subsubsection{Problem \label{par:Problem-monads-2}\ref{par:Problem-monads-2}}

Does a monad transformer exist for every exponential-polynomial monad
whose \lstinline!pure! and \lstinline!flatMap! methods are implemented
using purely functional code?

\subsubsection{Problem \label{par:Problem-monads-5-1}\ref{par:Problem-monads-5-1}}

We know that the monad transformers $T_{L}$ can be defined using
a \lstinline!swap! function for certain monads. Is this also the
case for any monad stacks built out of such monads? If each of the
monads $L_{1}$, $L_{2}$, ..., $L_{k}$ admits a transformer defined
via a lawful \lstinline!swap! function, will the monad $L_{1}\varangle L_{2}\varangle...\varangle L_{k}$
also admit a transformer with a \lstinline!swap! function?

\subsubsection{Problem \label{par:Problem-identity-natural-monad-morphism}\ref{par:Problem-identity-natural-monad-morphism}}

Are there any monadically natural monad morphisms $M\leadsto M$?
Equivalently, are there any natural transformations $\text{Id}^{\bullet}\leadsto\text{Id}^{\bullet}$
between identity functors in the category of monads, other than the
identity transformation?

We look for a monad morphism $\varepsilon^{M,A}:M^{A}\rightarrow M^{A}$
that is defined for all monads $M$ and is monadically natural in
the parameter $M$. So, the function $\varepsilon$ must satisfy the
following laws:
\begin{align*}
{\color{greenunder}\text{naturality law}:}\quad & \varepsilon^{M,A}\bef(f^{:A\rightarrow B})^{\uparrow M}=(f^{:A\rightarrow B})^{\uparrow M}\bef\varepsilon^{M,B}\quad,\\
{\color{greenunder}\text{monad morphism laws}:}\quad & \text{pu}_{M}\bef\varepsilon=\text{pu}_{M}\quad,\quad\quad\varepsilon^{\uparrow M}\bef\varepsilon\bef\text{ftn}_{M}=\text{ftn}_{M}\bef\varepsilon\quad,\\
{\color{greenunder}\text{monadic naturality law}:}\quad & \varepsilon^{M,A}\bef\phi^{:M^{:A}\rightarrow N^{A}}=\phi^{:M^{A}\rightarrow N^{A}}\bef\varepsilon^{N,A}\quad,
\end{align*}
where $f^{:A\rightarrow B}$ is an arbitrary function, $M$, $N$
are arbitrary monads, and $\phi:M\leadsto N$ is an arbitrary monad
morphism. The problem is either to prove that any such $\varepsilon$
must be an identity function, $\varepsilon=\text{id}^{:M^{A}\rightarrow M^{A}}$,
or to show an example of such $\varepsilon$ not equal to identity.\footnote{See discussion here: \texttt{\href{https://stackoverflow.com/questions/61444425/}{https://stackoverflow.com/questions/61444425/}}}

\subsubsection{Problem \label{par:Problem-monads-3}\ref{par:Problem-monads-3}}

\textbf{(a)} Are there any rigid functors that are not monads? 

\textbf{(b)} Are there any rigid functors that are not applicative?

\textbf{(c)} Is it true that any applicative rigid functor is a monad?

\subsubsection{Problem \label{par:Problem-monads-5-2}\ref{par:Problem-monads-5-2}}

Assume an arbitrary unknown monad $M$ and define $L^{A}\triangleq\bbnum 1+A\times M^{L^{A}}$.
(This is the \lstinline!List! monad\textsf{'}s transformer without the outer
layer of $M$.) Can one define a lawful monad instance for the functor
$L$? (See  Exercise~\ref{subsec:Exercise-effectful-list-not-monad}.)
